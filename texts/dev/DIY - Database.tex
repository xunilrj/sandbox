\documentclass[10pt,a4paper]{book}
\usepackage[latin1]{inputenc}
\usepackage{amsmath}
\usepackage{amsfonts}
\usepackage{amssymb}
\usepackage{graphicx}
\usepackage{listings}
\usepackage{xcolor}
\lstset{
	frame=tb, % draw a frame at the top and bottom of the code block
	tabsize=4, % tab space width
	showstringspaces=false, % don't mark spaces in strings
	numbers=left, % display line numbers on the left
	commentstyle=\color{green}, % comment color
	keywordstyle=\color{blue}, % keyword color
	stringstyle=\color{red} % string color
}
\author{Daniel Frederico Lins Leite}
\title{DIY - Database}
\begin{document}
	\chapter{REPL}
	
	\begin{lstlisting}[language=C++, caption={C++ code using listings}]
	#include <stdio.h>
	#include <stddef.h>
	#include <stdint.h>
	#include <stdlib.h>
	#include <string>
	#include <sstream>
	#include <iostream>
	#include <functional>
	#include <tuple
	\end{lstlisting}
	
	\begin{lstlisting}[language=C++, caption={C++ code using listings}]
	int main(int argc, char* argv[]) 
	{
		REPL repl;
		repl.run([](auto str, auto& ss) {
			if (str == ".exit") return -1;
			else 
			{
				ss << "Unrecognized command [" 
				   << str << "]";
				return 0;
			}
		});
		exit(EXIT_SUCCESS);
		return 0;
	}
	\end{lstlisting}
	
	\pagebreak
	
	\begin{lstlisting}[language=C++, caption={C++ code using listings}]
	using PREPLCallback = int(
		const std::string&,
		std::stringstream& ss);
	using REPLCallback = std::function<PREPLCallback>;
	class REPL
	{
	public:
		REPL() :
			in(std::cin),
			out(std::cout)
		{ }
		REPL(std::istream& in, std::ostream& out) :
			in(in),
			out(out)
		{ }
		
		void run(REPLCallback f) { ... }
	private:
		std::istream& in;
		std::ostream& out;
		
		std::string read_input() { ... }	
	};
	\end{lstlisting}
	
	\begin{lstlisting}[language=C++, caption={C++ code using listings}]
	void run(REPLCallback f)
	{
		std::stringstream ss;
		while (true) 
		{
			out << "db> ";
			
			ss.str("");
			ss.clear();
			
			auto str = read_input();
			auto status = f(str, ss);
			if (status > 0)
			return exit(status);
			else if (status < 0)
			break;
			
			out << ss.str() << std::endl;
		}
	}
	\end{lstlisting}
	
	\begin{lstlisting}[language=C++, caption={C++ code using listings}]
	std::string read_input() 
	{
		std::string str;
		std::getline(in, str);
		if (str.size() <= 0)
		{
			out << "Error reading input." << std::endl;
			exit(EXIT_FAILURE);
		}
		return str;
	}
	\end{lstlisting}
\end{document}