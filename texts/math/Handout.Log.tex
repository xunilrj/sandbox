\documentclass{article}
\usepackage[utf8]{inputenc}
\usepackage[english]{babel} 
\usepackage{amsmath,amssymb,amsthm}
\usepackage{hyperref}

\newtheorem{theorem}{Theorem}[section]
\theoremstyle{definition}
\newtheorem{definition}{Definition}[theorem]
\newtheorem{corollary}{Corollary}[theorem]
\newtheorem{lemma}[theorem]{Lemma}
\newcommand\numberthis{\stepcounter{equation}{1}\tag{\theequation}}

\title{Mathematics Handout - Logaritmics}
\author{Daniel Frederico Lins Leite}
\date{July 2016}

\begin{document}
\section{Introduction}
\begin{definition}[Log Definition]\label{definitions:log}
	\begin{align*}
		log_a b=c \\
		a^{log_a b}=a^c \\
		b=a^c \\
		log_a b=log_a a^c \\
		log_a b= c		
	\end{align*}
\end{definition}
\begin{theorem}[Logarithm Product Rule]\label{log:product.rule}
	\begin{align*}
	log_x{(A*B)}=log_x A + log_x B
	\end{align*}
	\begin{proof}
		\begin{align}			
			x^l = A \\
			log_x {x^l} = log_x A \\
			l = log_x A \label{logproductrulel} \\ 			
			\nonumber\\
			x^m = B \\
			log_x {x^m} = log_x B \\
			\label{logproductrulem} m = log_x B \\
			\nonumber\\
			x^n = A*B \\
			log_x{x^n} = log_x{(A*B)} \\
			\label{logproductrulen}n = log_x{(A*B)} \\
			\nonumber\\
			log_x{(A*B)} = n \\
			x^n = A*B \\
			x^n	= x^l*x^m \\
			x^n	= x^{l+m} \\
			n = l + m && \text{use (\ref{logproductrulel}) (\ref{logproductrulem}) 	(\ref{logproductrulen})}\\
			log_x{(A*B)} = log_x A + log_x B		
		\end{align}
	\end{proof}	
\end{theorem}
\begin{theorem}[Logarithm Power Rule]\label{log:power.rule}
	\begin{align*}
	log_x{A^B}=B*log_x(A)
	\end{align*}
	\begin{proof}
		\begin{align}
		log_x{(A^B)} \\
		log_x{(\prod_{n=1}^{B}A)}\\
		\sum_{n=1}^{B}(log_x{A}) && \text{use (\ref{log:product.rule})}\\
		log_x{A}*\sum_{n=1}^{B}1\\
		B*log_x{A}
		\end{align}
	\end{proof}
\end{theorem}
\end{document}