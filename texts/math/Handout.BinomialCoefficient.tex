\documentclass[]{article}
\usepackage{amsmath,amssymb,amsthm}
\usepackage{hyperref}

\begin{document}
	
	The notation $\binom{N}{K}$ for the binomial coefficient was introduced by Andreas Freiherr von Ettingshausen in his 1826 work "Die combinatorische Analysis".
	
	It appears to have become the de facto standard in recent years.\\
	\\
	Definition:
\begin{align}
	\binom{n}{k} &= \frac {n!}{k!(n-k)!}\\
\end{align}
	Recursive Definition:
\begin{align}
	\binom{n}{k} &= \binom{n - 1}{k - 1} + \binom{n - 1}{k}\\ 
	\binom{n}{0} &= 1
\end{align}

Identities 
\begin{align}
	\binom{n}{k} &= \frac {n}{k} \binom{n-1}{k-1}
\end{align}
proof
\begin{align}
	\binom{n}{k} &= \frac {n!}{k!(n-k)!}\\
	&= \frac{n*(n-1)!}{k*(k-1)!(n-k)!}\\
	u &= n-1\\
	z &= k-1\\
	u-z &= (n-1)-(k-1)\\
	&= n-1-k+1\\
	&= n-k\\
	&= \frac{n}{k} * \frac {u!}{z!(u-z)!}\\
	&= \frac{n}{k} * \binom{u}{z}\\
	&= \frac{n}{k} * \binom{n-1}{k-1}
\end{align}


See More:
\url{https://www.khanacademy.org/math/precalculus/prob-comb/combinations/v/introduction-to-combinations}
\url{https://www.khanacademy.org/math/algebra2/polynomial-functions/binomial-theorem/v/binomial-theorem}\\


\end{document}