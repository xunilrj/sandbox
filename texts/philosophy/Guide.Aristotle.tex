\documentclass[10pt,a4paper]{book}
\usepackage[latin1]{inputenc}
\usepackage{amsmath}
\usepackage{amsfonts}
\usepackage{amssymb}
\usepackage{graphicx}
\author{Daniel Frederico Lins Leite}
\title{Guide to Aristotle}
\begin{document}
	\chapter{Modern Scholars}
	
	\section{Giovanni Reale}
	
	
	
	\section{Jonathan Barnes}
	
	Books\\
	Aristotle: A very short introduction\\
	\\
	No man before him had contributed so much to learning. No man after him might aspire to rival his achievements.\\
	\\
	In one of his later works, the 	Nicomachean Ethics, Aristotle argues that ?happiness? ? that state of 	mind in which men realize themselves and flourish best ? consists in a 	life of intellectual activity. Is not such a life too godlike for mere 	mortals to sustain? No; for ?we must not listen to those who urge us to 	think human thoughts since we are human, and mortal thoughts since 	we are mortal; rather, we should as far as possible immortalize 	ourselves and do all we can to live by the finest element in us ? for if in bulk it is small, in power and worth it is far greater than anything else?.
	\\
	A good way of reading him is this: Take up 	a treatise, think of it as a set of lecture notes, and imagine that you now have to lecture from them. You must expand and illustrate the argument, and you must make the transitions clear; you will probably decide to relegate certain paragraphs to footnotes, or reserve them for another time and another lecture;
	\\
	The Cambrigde Companion to Aristotle\\
	The Cambridge History of Hellenistic Philosophy\\
	The Complete Works of Aristotle
	
	\chapter{Timeline}
	
	Spring 322: Retire to Chalcis (Euboea Island)
\end{document}