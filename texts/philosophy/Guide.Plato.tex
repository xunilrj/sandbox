\documentclass[10pt,a4paper]{article}
\usepackage[latin1]{inputenc}
\usepackage{amsmath}
\usepackage{amsfonts}
\usepackage{amssymb}
\usepackage{graphicx}
\author{Daniel Frederico Lins Leite}
\begin{document}
	\maketitle
	\begin{chapter}
			
	\end{chapter}

Laws

[Eric Voegelin - Order and History - Plato and Aristotle]

The Laws has only one major, external incision?
the one that separates the first three books from the subsequent
nine. And this incision serves precisely the purpose of articulation
through dominant motifs inasmuch as the first three books marshal
the motifs, while Books IV?XII apply them to the exposition of the
nomoi of the polis. Within the nine books the subject matter itself
is distributed in a roughly systematic sequence from the choice of
site and population for the polis, through social institutions, mag-
istracies, education, and festivals, to civil and criminal law?but
this systematic order is subordinated to the order of the dominant
motifs.

The high points of this internal organization are to be found
in Books VII (Education) and X (Religion) with the recurrence of
the principal motif of Book I, that is, with the symbol of the God
who plays the game of order and history with man as his puppet.

This
Summa of Greek life embraces in its amplitude the consequences
of the Trojan War and the Doric invasion, it analyzes the failure
of the Doric military kingship and the horrors of Athenian the-
atrocracy, it reflects on the effects of harem education on Persian
kings and on the preservation of art styles in Egypt, on the con-
sequences of enlightenment and the inquisitorial enforcement of a
creed, on the ethos of musical scales as well as on the undesirability
of fishing as a sport

The Laws develops fully the religious position of Plato; the polis
whosefoundationandorganizationisdescribedmaybecharacterized
as a theocratic state. In the liberal era a work of this kind could only
arouse grave misgivings among scholars for whom the separation
of church and state was a fundamental dogma, and for whom a
theory of politics had to be defined in terms of the secular state.

The second of these elementary misconceptions crystallizes in
the formula that the Laws is a treatise on ?jurisprudence.? The as-
sumption is not entirely wrong. The Laws, indeed, contains sections
that have to be classified as jurisprudence in the modern meaning
of the word

The error is in part induced by the translation of
the title Nomoi as Laws with the implication that the laws of Plato
mean laws in the sense in which the word is used in modern legal
theory. Plato?s nomos, however, is deeply imbedded in the myth of
nature and has an amplitude of meaning that embraces the cosmic
order, festival rites, and musical forms. The assumption that the
Laws is a treatise on ?jurisprudence? ignores this range of meaning
and inevitably destroys the essence of Plato?s thought.

A third group of misconceptions touches a more complex problem.
ItconcernsPlato?sdesignationoftheLawsasthesecond-bestplanfor
a polis. We hear that Plato had developed in the Republic his plan for
theidealstate,thathisprojectwasutterlyunrealistic,thathehimself
became convinced of its Utopian character, and, finally, that in the
Laws he developed an ideal that took into account at least some of
the exigencies of human nature. Moreover, the Laws is distinguished
from the Republic by its recognition of historical traditions and
customs as well as of the necessity of legal institutions. While the
Republic still envisaged something like a dictatorial government
without laws, the Laws embodies the more mature conception of a
constitutionalgovernmentunderthelaws.

The plan of a second-best polis seems to imply a transition from
the ?ideal? of dictatorship by the philosopher-king to the ?ideal? of
a government by law with constitutional consent of the people. In
thisinstancewehavetounravelawholeseriesofmisunderstandings.


The Republic is
written under the assumption that the ruling stratum of the polis
will consist of persons in whose souls the order of the idea can
become reality so fully that they, by their very existence, will be the
permanent source of order in the polis.  the Laws is written under the
assumption that the free citizenry will consist of persons who can
be habituated to the life of Arete under proper guidance, but who are
unabletodevelopthesourceoforderexistentiallyinthemselvesand,
therefore, need the constant persuasion of the prooemia as well as
the sanctions of the law, in order to keep them on the narrow path

[Herbert Marcuse - The Af?rmative Character of Culture ]

 Platonic philosophy still contended with the social order of commercial Athens. Plato?s idealism is interlaced with motifs of social criticism. What appears as facticity from the standpoint of the Ideas is the material world in which men and things encounter one another as commodities. The just order of the soul is destroyed by 
 
 the passion for wealth which leaves a man not a moment of leisure to attend to anything beyond his personal fortunes. So long as a citizen?s whole soul is wrapped up in these, he cannot give a thought to anything but the day?s takings.7 
 
 And the authentic, basic demand of idealism is that this material world be transformed and improved in accordance with the truths yielded by knowledge of the Ideas. Plato?s answer to this demand is his program for a reorganization of society. This program reveals what Plato sees as the root of evil. He demands, for the ruling strata, the abolition of private property (even in women and children) and the prohibition of trade. This same program, however, tries to root the contradictions of class society in the depths of human nature, thereby perpetuating them.
 
Critias
Timaeus
The Republic 
Menexenus 
Ion 
Hipias Mayor 
Lesser Hippias 
Meno 
Gorgias 
Protagoras 
Lysis 
Laches 
Charmides 
Lovers 
Theages 
Hipparchus 
Alcibiades 2 
Alcibiades I 
Phaedrus 
Symposium 
Philebus 
Parmenides 
Statesman 
Phaedo 
Sophist (Annotated) 
Theaetetus 
Cratylus 
Euthyphro 

Socrates: A Man for Our Times Johnson, Paul 
Plat�o  Reale, Giovanni
Storia della filosofia greca e romana vol. 3 - Platone e l'Accademia antica Reale, Giovanni
Toward a New Interpretation of Plato Reale, Giovanni
Plato and Aristotle Voegelin, Eric 

Good Articles:
"I read two dialogues of Plato, the ?Timaeus? and the ?Parmenides.? These are among Plato?s longer and more difficult dialogues ? the first about creating the world, and the second about the One."
http://www.crisismagazine.com/2011/what-plato-advises

\end{document}