\documentclass{article}
\usepackage[latin1]{inputenc}
\usepackage{amsmath}
\usepackage{amsfonts}
\usepackage{amssymb}
\usepackage{graphicx}
\usepackage{hyperref}
\usepackage{pgfplots}
\begin{document}
	Exercise 1.4.72\\
	Figure out what the following MATLAB code does. 
	\begin{verbatim}
	n = 1000; 
	for jay = 1:4 
		if jay > 1; oldtime = time; end;
		M = randn(n); 
		A = M'*M; 
		t = cputime; 
		R = chol(A); 
		matrixsize = n
		time = cputime - t
		if time == 0; time = 1; end;
		if jay > 1; ratio = time/oldtime; end;
		n = 2*n; 
	end 	
	\end{verbatim}		
	If you put these instructions into a file named, say, zap.m, you can run the program by 
	typing zap from the MATLAB command line. The functions randn, cputime, 
	and chol are built-in MATLAB functions, so you can learn about them by typing 
	help randn, etc. You might find it useful to type more on before typing help 
	{topic}.\\ 
	(a) Does the code produce reasonable values of ratio when you run it? What 
	value would you expect in theory? Depending on the speed of the machine on 
	which you are running MATLAB, you may want to adjust $n$ or the number of 
	times the j ay loop is executed.\\ 
	(b) Why is the execution time of this code (i.e. the time you sit waiting for the 
	answers) so much longer than the times that are printed out?\\ 
	\\
	Execution:
	\begin{verbatim}
	> octave .\exercise.01.004.0072.m
	matrixsize =  1000
	time = 0
	matrixsize =  2000
	time =  1.0625
	matrixsize =  4000
	time =  10.125
	matrixsize =  8000
	time =  81.406
	\end{verbatim}
	Answers:\\
	\\
	(a) First we need to adjust $n$, because it is instantaneous to matrices smaller than 1000.\\
	By Proposition 1.4.24, the Cholesky decomposition should increase by $O(n^3)$.\\
	\\
	Proposition 1.4.24 Cholesky's algorithm 1.4.17 applied to an $n \times n$ matrix performs about $\frac{n^3}{3}$ flops.\\ 
	\\
	Which was exactly what happedned:\\
	\begin{verbatim}
	> 10.125 / 1.0625
	9.52941176470588
	> 81.406 / 10.150
	8.02029556650246
	\end{verbatim}
	(b) Because we are not counting the time spent generating $M$ and the time multiplying $M^TM$. Time needed to guarantee that we will pass a valid matrix to the "chol" method.
\end{document}