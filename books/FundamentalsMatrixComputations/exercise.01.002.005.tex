\documentclass{article}
\usepackage[latin1]{inputenc}
\usepackage{amsmath}
\usepackage{amsfonts}
\usepackage{amssymb}
\usepackage{graphicx}
    \begin{document}
    	If $A^{-1}$ exists then $det(A) \neq 0$.\\
    	To solve this we need.\\
    	\begin{align*}
    		det(A*B) = det(A)*det(B)\\
    		det(A^{-1})=\frac{1}{det(A)}
    	\end{align*}
       	\begin{align*}
       	Ax=b\\
       	A^{-1}Ax=A^{-1}b\\
		Ix=A^{-1}b\\
		x=A^{-1}b\\
		det(x)=det(A^{-1}b)\\
		det(x)=det(A^{-1})det(b)\\
		det(x)=\frac{det(b)}{det(A)}
		\end{align*}
		And this last step is definided in all points that $det(A) \neq 0$.
    \end{document}