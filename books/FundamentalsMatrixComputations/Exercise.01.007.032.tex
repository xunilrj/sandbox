\documentclass[10pt,a4paper]{article}
\usepackage[latin1]{inputenc}
\usepackage{amsmath}
\usepackage{amsfonts}
\usepackage{amssymb}
\usepackage{graphicx}
\usepackage{hyperref}
\author{Daniel Frederico Lins Leite}
\begin{document}
	Proposition 1.4.55\\
	If $A$ and $X$ are $n \times n$, $A$ is positive definite and $X$ is nonsingular, then the matrix $B = X^TAX$ is also positive definite.\\
	\\
	Theorem 1.7.31\\
	Let A be positive definite. Then A can be expressed in exactly one 
	way as a product $A = LDL^T$, such that L is unit lower triangular, and D is a 
	diagonal matrix whose main-diagonal entries are positive. \\
	\\
	Exercise 1.7.32\\
	Prove Theorem 1.7.31.\\
	\\
	To this end it suffices to show that A is positive definite if and only if the main-diagonal entries of D are positive. See Proposition 1.4.55.\\
	\\
	Proof:\\
	\\
	Part 1: If A is "Positive Definite" then main-diagonal entries of D are positive.
	\begin{align*}
		X^TAX = B\\
		X^{-T}X^TAXX^{-1} = X^{-T}BX^{-1}\\
		A = X^{-T}BX^{-1}\\
		\\
		L = X^{-1}\\
		\\
		A = L^TBL
	\end{align*}
	By the uniqueness of the factorization there is only one factorization of the form $A = L^TDL$. Which makes $B=D$. But from the "Positive Definiteness" of the matrix, $B$, and off course $D$ are "Positive Definite".
	Which implies that the main-diagonal entries of the matrix are positive $\square$.
	\\
	\\
	Part 2: If main-diagonal entries of D are positive then A is "Positive Definite"\\	
	\\
	If D is a diagonal matrix with  main-diagonal entries then D is "Positive Definite".
	\begin{align*}			
		B &= X^TDX\\
		\\
		L &= X\\
		\\
		B &= L^TDL\\
	\end{align*}	
	Again, by uniqueness of factorization, $B=A$, and given that $B$ is "Positive Definite", we conclude that $A$ is "Positive Definite". $\square$
\end{document}




