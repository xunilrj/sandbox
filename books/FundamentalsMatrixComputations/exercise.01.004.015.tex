\documentclass{article}
\usepackage[latin1]{inputenc}
\usepackage{amsmath}
\usepackage{amsfonts}
\usepackage{amssymb}
\usepackage{graphicx}
\usepackage{systeme}
\usepackage{hyperref}
\begin{document}
	Exercise 1.4.15\\
	Let\\
	\begin{align*}
	A &= \begin{bmatrix}
		4 & 0\\
		0 & 9\\
		\end{bmatrix}
	\end{align*}
	(a) Prove that A is positive definite,\\
	(b) Calculate the Cholesky factor of $A$,\\
	(c) Find three other upper triangular matrices $R$ such that $A = R^TR$,\\
	(d) Let A be any $n \times n$ positive definite matrix. How many upper-triangular matrices $R$ such that $A = R^TR$ are there?\\
	\\
	Based on the work of \url{https://en.wikipedia.org/wiki/Andr\%C3\%A9-Louis_Cholesky}\\
	\\
	Proofs:\\
	(a)\\
	\begin{align*}
	x^TAx &= \begin{bmatrix}
	x_{11} & x_{21}\\
	\end{bmatrix} * \begin{bmatrix}
	4 & 0\\
	0 & 9\\
	\end{bmatrix} * \begin{bmatrix}
	x_{11}\\
	x_{21}\\
	\end{bmatrix}\\
	&= \begin{bmatrix}
	x_{11}*4 + x_{21}*0 & x_{11}*0 + x_{21}*9\\
	\end{bmatrix} * \begin{bmatrix}
	x_{11}\\
	x_{21}\\
	\end{bmatrix}\\
	&= \begin{bmatrix}
	x_{11}*4 & x_{21}*9\\
	\end{bmatrix} * \begin{bmatrix}
	x_{11}\\
	x_{21}\\
	\end{bmatrix}\\
	&= \begin{bmatrix}
	x_{11}*4*x_{11} & x_{21}*9*x_{21}\\
	\end{bmatrix}\\	
	&= \begin{bmatrix}
	4x_{11}^2 & 9x_{21}^2\\
	\end{bmatrix} \text{ necessarily} > 0\\	
	\end{align*}
	\\
	(b)\\
	\begin{align*}
		R &= \begin{bmatrix}
			r_{11} & r_{12}\\
			0 & r_{22}\\
			\end{bmatrix}\\
		A &= \begin{bmatrix}
		4 & 0\\
		0 & 9\\
		\end{bmatrix}\\
		R^TR &= A\\
		\begin{bmatrix}
		r_{11} & 0\\
		r_{12} & r_{22}\\
		\end{bmatrix} \begin{bmatrix}
		r_{11} & r_{12}\\
		0 & r_{22}\\
		\end{bmatrix} &= \begin{bmatrix}
		4 & 0\\
		0 & 9\\
		\end{bmatrix}\\
		\begin{bmatrix}
		r_{11}*r_{11}+0*0 & r_{11}*r_{12} + 0*r_{22}\\
		r_{12}*r_{11}+r_{22}*0 & r_{12}*r_{12} + r_{22}*r_{22}\\
		\end{bmatrix} &= \begin{bmatrix}
		4 & 0\\
		0 & 9\\
		\end{bmatrix}\\
		\\
		r_{11}*r_{11}+0*0 = 4\\
		r_{11}*r_{12} + 0*r_{22} = 0\\
		r_{12}*r_{11}+r_{22}*0 = 0\\
		r_{12}*r_{12} + r_{22}*r_{22} = 9\\
		\\
		r_{11}^2 = 4\\
		r_{11}*r_{12} = 0\\
		r_{12}*r_{11} = 0\\
		r_{12}^2 + r_{22}^2 = 9\\
		\\
		r_{11} = 2\\
		r_{11}*r_{12} = 0\\
		r_{12}*r_{11} = 0\\
		r_{12}^2 + r_{22}^2 = 9\\
		\\
		r_{11} = 2\\
		2*r_{12} = 0\\
		r_{12}*2 = 0\\
		r_{12}^2 + r_{22}^2 = 9\\	
	\end{align*}
	\begin{align*}
		r_{11} = 2\\
		r_{12} = 0\\
		r_{12} = 0\\
		r_{12}^2 + r_{22}^2 = 9\\
\\		
		r_{11} = 2\\
		r_{12} = 0\\
		r_{12} = 0\\
		0^2 + r_{22}^2 = 9\\
		\\
		r_{11} = 2\\
		r_{12} = 0\\
		r_{12} = 0\\
		r_{22}^2 = 9\\
		\\
		r_{11} = 2\\
		r_{12} = 0\\
		r_{12} = 0\\
		r_{22} = 3\\
		\\
		R = \begin{bmatrix}
			2 & 0\\
			0 & 3\\
		\end{bmatrix}\\
	\end{align*}
	\\
	(c)\\
	\\
	The Cholesky Factor must have all its diagonal entries greater than zero. That is what allowed us in (b) to choose $2$ and $3$ instead of $-2$ and $-3$ when finding the squares roots.\\
	But we can still find other matrices that satisfy the proposition $A=R^TR$ using this negative roots.\\
	\begin{align*}
		R = \begin{bmatrix}
		-2 & 0\\
		0 & 3\\
		\end{bmatrix}\\
		\\
		R = \begin{bmatrix}
		2 & 0\\
		0 & -3\\
		\end{bmatrix}\\
		R = \begin{bmatrix}
		-2 & 0\\
		\color{red}{0} & -3\\
		\end{bmatrix}
	\end{align*}
	All of them when "squared" generate the ogirinal matrix.\\
	\\
	(d) We can genralize this idea. For $A= n \times n$ matrix we can always find $N$ matrices where each one of them came from a permutation of the "signals". Given that we have $n$ elements in the diagonal and each of then can have 2 possible signals, we have $2^n$ possible matrices.
\end{document}