\documentclass[12pt]{article}
\usepackage{amsmath}
\usepackage{pdfpages}
\usepackage{lscape}
\title{2D Graphics with Asymptote}
\author{The Asymptote Project}
\newcommand{\insertrep}[1]{%
	\hspace*{-2.4cm}
	\fbox{\includegraphics[page=1,scale=1]{#1}}
}
\begin{document}
	Exercise 10:\\
	Which point of the cube is $i+j$. Which point is the vector sum of $i=(1,0,0)$ and $j=(0,1,0)$ and $k=(0,0,1)$? Describe all points $(x,y,z)$ in the cube.\\
	We will generate random points using three random values $a$, $b$ and $c$ as:
	$$a*A+b*B+c*C$$
	In this figure we try some random linear combinations using unbounded values for $a$, $b$, $c$. We see that points are spread both inside and outside the unit cube.\\
	\begin{center}
		\insertrep{exercise01010a.pdf}
	\end{center}
	In the second attemp we try to normalize the three weights as their sum in 1.\\
	\begin{align*}
		a + b + c = 1\\
		\\
		a = \frac{a}{a+b+c}\\
		b = \frac{b}{a+b+c}\\
		c = \frac{c}{a+b+c}\\				
	\end{align*}
	We can see that altought all random points fall inside the but they are not as spread as possible. We have a filling that they do no span all possible cubes. Which makes sense given that as $x$ and $y$ approaches $0.5$, $z$ converges to zero.	
	\begin{center}
		\insertrep{exercise01010b1.pdf}
	\end{center}
	This figure is the same as above but with more random points.
	\begin{center}
		\insertrep{exercise01010b2.pdf}
	\end{center}
	\newpage
	And the following figure is when we bound the random points to the interval $[0,1]$.
	\begin{center}
		\insertrep{exercise01010c.pdf}
	\end{center}	
\end{document}