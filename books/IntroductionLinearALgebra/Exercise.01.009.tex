\documentclass[12pt]{article}
\usepackage{amsmath}
\usepackage{pdfpages}
\usepackage{lscape}
\title{2D Graphics with Asymptote}
\author{The Asymptote Project}
\newcommand{\insertrep}[1]{%
	\hspace*{-2.4cm}
	\fbox{\includegraphics[page=1,scale=0.6]{#1}}
}
\begin{document}
	Exercise 9:\\
	If three corners of a parallelogram are $(1,1)$, $(4,2)$ and $(1,3)$ what are all three of the possible fourth corners? Draw two of them.\\
	\\
	This question is a little bit tricky. If you have three points of a parallelogram, we definitely know the forth point. But with three points and no specific order, you actually have three possible orders, with means that you have actually three possible parallelograms.\\
	For example, we have $A$, $B$ and $C$, so wue actually have "partial-parallelograms": $ABC$, $ACB$, and $CAB$. We actually have six permutations: $3!=3*2*1=6$, but some permutation are actually the same. For example: $BAC$ and $CAB$, $ABC$ and $CBA$, $BCA$ and $ACB$. If you look carefully, you will see that the order of the first and the third point does not matter in this case, just the "middle" point matters.\\
	We can calculate one of the points using the pallelogram rule that its opposing sides are equal. In the parallelogram $ABCD$, we have that $AC=BD$. Others permutations will generate the other points.
	\begin{align*}
		C-A=D-B\\
		\\
		\begin{bmatrix}1\\3\end{bmatrix}-\begin{bmatrix}1\\1\end{bmatrix} = \begin{bmatrix}d_1\\d_2\end{bmatrix}-\begin{bmatrix}4\\2\end{bmatrix}\\
		d_1 - 4 = 1 - 1\\
		d_2 - 2 = 3 - 1\\
		\\
		d_1 - 4 = 0\\
		d_2 - 2 = 2\\
		\\
		d_1 = 4\\
		d_2 = 4\\
	\end{align*}
	\begin{landscape}
		\begin{center}
			\insertrep{exercise01009.pdf}
		\end{center}
	\end{landscape}
\end{document}