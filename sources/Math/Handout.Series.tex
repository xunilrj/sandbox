\documentclass{article}
\usepackage[utf8]{inputenc}
\usepackage[english]{babel} 
\usepackage{amsmath,amssymb,amsthm}
\newtheorem{theorem}{Theorem}[section]
\newtheorem{corollary}{Corollary}[theorem]
\newtheorem{lemma}[theorem]{Lemma}
\title{Mathematics Handout - Logaritmics}
\author{Daniel Frederico Lins Leite}
\date{July 2016}
\begin{document}
\section{Introduction}
\maketitle

\[{x_n}\] is a sequence of real numbers
Th sequence converge if:
$$\exists a \in Re$$
$$\exists \epsilon \in R such that \epsilon > 0$$
$$\exists N_a \in I such that N_a > 0$$
$$\forall n > N_a, |a - x_n| < \epsilon$$

if for example exists another 
$$\exists b \in Re$$ such that
$$\exists N_b \in I such that N_b > 0$$
$$\forall n > N_b, |b - x_n| < \epsilon$$

then 

$$|a-b| = |a - x_n + x_n - b|$$

Given that the $$R^n$$, the space of the elements of the sequence is a normed vector space, we can apply the Triangle Inequality. So

$$|a - x_n + x_n - b| <= |a - x_n| + |x_n - b|$$

given that $$|x_n - b|$$ is a even function
$$f(x) = f(-x)$$ we have that
$$|x_n - b| = -1 * |x_n - b| = |-x_n + b| = |b - x_x|$$

so we have

$$|a - x_n| + |x_n - b| < |a - x_n| + |b - x_n| <= \epsilon + \epsilon <= 2\epsilon$$

So

$$ 0 <= |a - b| <= 2\epsilon$$ 

So $|a-b|$ is squeezed between this two function. Given that this is true for every $\epsilon$, if $\epsilon -> 0$
we will have that:
$$|a-b| = 0$$ and $$a=b$$

The value a is called the limit of the sequence.

The 

$$S = \sum_{i=0}^{\inf} {x_i}$$ is called the series associated with se sequence.

The partial sum is

$$S_n = \sum_{i=0}^{n} {x_i}$$

So

${S_n}$ is also a sequence. In the same way it can converge and have a limit.
If the sequence converge, we say that the series converge and that the limit is

$$limit of {S_n} =  \sum_{i=0}^{\inf} {x_i}$$

If we have two convergent series:

$$S_1 = \sum_{i=0}^{\inf} {a_i}$$
$$S_2 = \sum_{i=0}^{\inf} {b_i}$$

then

$$\sum_{i=0}^{\inf} {(a_i + b_i)} = \sum_{i=0}^{\inf} {a_i} + \sum_{i=0}^{\inf} {b_i}$$






\end{document}